\section{D. Irga B. Naufal Fakhri}
\subsection{Pemahaman Teori}
\begin{enumerate}
\item Fungsi

Fungsi adalah blok blok kode yang teroorganisir yang dapat digunakan kembali didalam program yang digunakan untuk melakukan suatu perintah yang telah diberikan.
untuk membuat fungsi kita harus menggunakan def kemudian nama fungsinya dan (variable)nya diakhiri oleh tanda :
\lstinputlisting[caption=Contoh kode fungsi inputan ke fungsi., firstline=296, lastline=301]{src/1174066.py}
Fungsi juga berguna untuk melemparkan variable contohnya
\lstinputlisting[caption=Contoh kode fungsi outputan ke fungsi., firstline=303, lastline=308]{src/1174066.py}

\item Paket(Package) atau Libary

Paket atau yang biasa disebut dengan library adalah kumpulan kode-kode fungsi atau method pada python yang dapat dipanggil kedalam program python yang kita buat. Package berada di file terpisah dari main program
cara memanggil package: Pastikan file package ada didalam folder yang sama lalu ditambah import dengan nama filenya tanpa extensi (.py)
\lstinputlisting[caption=Contoh import package atau library., firstline=311, lastline=314]{src/1174066.py}

\item Kelas (Class), Objek (Object), Atribut (Attribute), dan Method

Kelas(Class) adalah sebuah blueprint(cetakan) dari sebuah objek.
Objek(Object) adalah hasil cetakan dari sebuah kelas(class).
Atribut(Attribute) adalah nilai data yang ada didalam sebuah object.
Method adalah sesuatu yang bisa dilakukan oleh object.

\lstinputlisting[caption=Contoh import package atau library., firstline=316, lastline=328]{src/1174066.py}

\item Cara memanggil library dari instansiasi

Cara memanggilnya:
\begin{itemize}
	\item Pertama kita import filenya
	\item kemudian buat variablenya jika menggunakan variable untuk menampung data
	\item Kemudian panggil nama classnya(file) dan panggil fungsinya
	\item Kemudian menggunakan perintah print untuk menampilkan data
\end{itemize}
\lstinputlisting[caption=Contoh package atau library., firstline=6, lastline=9]{src/fungsi_1174066.py} 
\lstinputlisting[caption=Contoh import package atau library., firstline=331, lastline=336]{src/1174066.py}

\item  Contoh pemakaian paket dengan perintah from kalkulator import Penambahan 

Pemakaian package(paket) dengan perintah from namafilenya import berfungsi untuk memanggil fungsi dari nama filenya
\lstinputlisting[caption=Contoh import package atau library., firstline=339, lastline=344]{src/1174066.py}

\item Jelaskan dengan contoh kode, pemakaian paket fungsi didalam folder

Jika file paket ada didalam folder maka kita harus menambahkan lokasi filenya ada didalam folder apa dengan cara menggunakan namafolder.namafile
\lstinputlisting[caption=Contoh import package atau library didalam folder., firstline=346, lastline=351]{src/1174066.py}

\item Jelaskan dengan contoh kode, pemakaian paket fungsi didalam folder

Jika file paket ada didalam folder maka kita harus menambahkan lokasi filenya ada didalam folder apa dengan cara menggunakan namafolder.namafile
\lstinputlisting[caption=Contoh import package atau library didalam folder., firstline=346, lastline=351]{src/1174066.py}
\end{enumerate}

\subsection{Keterampilan Pemograman}
\begin{enumerate}
\item Jawaban nomor 1
\lstinputlisting[firstline=355, lastline=373]{src/1174066.py}

\item Jawaban nomor 2
\lstinputlisting[firstline=377, lastline=384]{src/1174066.py}

\item Jawaban nomor 3
\lstinputlisting[firstline=386, lastline=394]{src/1174066.py}

\item Jawaban nomor 4
\lstinputlisting[firstline=396, lastline=400]{src/1174066.py}

\item Jawaban nomor 5
\lstinputlisting[firstline=402, lastline=405]{src/1174066.py}

\item Jawaban nomor 6
\lstinputlisting[firstline=410, lastline=418]{src/1174066.py}

\item Jawaban nomor 7
\lstinputlisting[firstline=420, lastline=428]{src/1174066.py}

\item Jawaban nomor 8
\lstinputlisting[firstline=431, lastline=439]{src/1174066.py}

\item Jawaban nomor 9
\lstinputlisting[firstline=441, lastline=448]{src/1174066.py}

\item Jawaban nomor 10
\lstinputlisting[firstline=451, lastline=468]{src/1174066.py}

\item Jawaban nomor 11
\lstinputlisting[firstline=8, lastline=20]{src/main_1174066.py}

\item Jawaban nomor 12
\lstinputlisting[firstline=23, lastline=37]{src/main_1174066.py}
\end{enumerate}

\subsection{Ketrampilan Penanganan Error}
\begin{itemize}
\item Syntax Errors

Syntax Errors adalah kesalahan pada penulisan syntax atau kode. Solusinya adalah memperbaiki penulisan syntax atau kode

\item Zero Division Error

ZeroDivisonError adalah exceptions yang terjadi saat eksekusi program menghasilkan perhitungan matematika pembagian dengan angka nol (0). Solusinya adalah tidak membagi suatu yang hasilnya nol.

\item Name Error

NameError adalah exception saat kode melakukan eksekusi terhadap local name atau global name yang tidak terdefinisi atau tidak ada. Solusinya adalah memastikan variabel atau function yang akan dipanggil ada didalam program atau tidak salah mengetikannya.

\item Type Error

TypeError adalah exception saat melakukan eksekusi terhadap suatu operasi atau fungsi dengan type object yang tidak sesuai. Solusinya adalah mengkoversi varibelnya sesuai dengan tipe data sesuai dengan yang akan digunakan.
\end{itemize}
\lstinputlisting[firstline=23, lastline=37]{src/main_1174066.py}
%%%%%%%%%%%%%%%%%%%%%%%%%%%%%%%%%%%%%%%%%%%%%%%%%%%%%%%%%%%%%%
\section{Advent Nopele Olansi Damiahan Sihite}
\subsubsection{Pemahanan Teori}
\begin{enumerate}
    \item Apa itu fungsi, inputan fungsi dan kembalian fungsi dengan contoh kode program
    lainnya.
    Fungsi adalah bagian dari program yang dapat digunakan ulang.
    Berikut merupakan contoh fungsi dan cara pemanggilannya
    \lstinputlisting[firstline=124, lastline=127]{src/1174089.py}

    Fungsi dapat membaca parameter, parameter adalah nilai yang disediakan kepada fungsi, dimana nilai ini akan menentukan output yang akan dihasilkan fungsi.
    \lstinputlisting[firstline=129, lastline=132]{src/1174089.py}

    Statemen return digunakan untuk keluar dari fungsi. Kita juga dapat menspesifikasikan nilai kembalian.
    \lstinputlisting[firstline=134, lastline=141]{src/1174089.py}

    \item Apa itu paket dan cara pemanggilan paket atau library dengan contoh kode
    program lainnya.
    Untuk memudahkan dalam pemanggilan fungsi yang di butuhkan, agar dapat dipanggil berulang.
    Cara pemanggilannya
    \lstinputlisting[firstline=143, lastline=144]{src/1174089.py}

    \item Jelaskan Apa itu kelas, apa itu objek, apa itu atribut, apa itu method dan
    contoh kode program lainnya masing-masing.
    kelas merupakan sebuah blueprint yang mepresentasikan objek.
    objek adalah hasil cetakan dadri sebuah kelas.
    method adalah suatu upaya yang digunakan oleh object.
    \lstinputlisting[firstline=146, lastline=168]{src/1174089.py}

    \item Jelaskan cara pemanggikan library kelas dari instansiasi dan pemakaiannya den-
    gan contoh program lainnya.
    Cara Pemanggilanya 
    \begin{itemize}
        \item pertama import terlebih dahulu filenya.
        \item kemudian buat variabel untuk menampung datanya
        \item setelah itu panggil nama classnya dan panggil methodnya
        \item Gunakan perintah print untuk menampilkan hasilnya

    \end{itemize}
    \lstinputlisting[firstline=170, lastline=175]{src/1174089.py}

    \item Jelaskan dengan contoh pemakaian paket dengan perintah from kalkulator im-
    port Penambahan disertai dengan contoh kode lainnya.
    Penggunaan paket from namafile import, itu berfungsi untuk memanggil file dan fungsinya
    \lstinputlisting[firstline=143, lastline=144]{src/1174089.py}

    \item Jelaskan dengan contoh kodenya, pemakaian paket fungsi apabila le library
    ada di dalam folder.
    Pemakaian paket adalah perkumpulan fungsi-fungsi. contoh kodenya adalah sebagai berikut :

    \item Jelaskan dengan contoh kodenya, pemakaian paket kelas apabila le library ada
    di dalam folder.
    \lstinputlisting[firstline=184, lastline=184]{src/1174089.py}

\end{enumerate}
\subsubsection{Ketrampilan Pemrograman}
\begin{enumerate}
    \item Buatlah fungsi dengan inputan variabel NPM, dan melakukan print luaran huruf
    yang dirangkai dari tanda bintang, pagar atau plus dari NPM kita. Tanda
    bintang untuk NPM mod 3=0, tanda pagar untuk NPM mod 3 =1, tanda plus
    untuk NPM mod3=2.
    \lstinputlisting[firstline=184, lastline=234]{src/1174089.py}

    \item Buatlah fungsi dengan inputan variabel berupa NPM. kemudian dengan meng-
    gunakan perulangan mengeluarkan print output sebanyak dua dijit belakang
    NPM.
    \lstinputlisting[firstline=237, lastline=243]{src/1174089.py}

    \item Buatlah fungsi dengan dengan input variabel string bernama NPM dan beri
    luaran output dengan perulangan berupa tiga karakter belakang dari NPM se-
    banyak penjumlahan tiga dijit tersebut.
    \lstinputlisting[firstline=245, lastline=255]{src/1174089.py}

    \item Buatlah fungsi hello word dengan input variabel string bernama NPM dan
    beri luaran output berupa digit ketiga dari belakang dari variabel NPM meng-
    gunakan akses langsung manipulasi string pada baris ketiga dari variabel NPM.
    \lstinputlisting[firstline=257, lastline=263]{src/1174089.py}

    \item buat fungsi program dengan input variabel NPM dan melakukan print nomor npm satu persatu kebawah.
    \lstinputlisting[firstline=265, lastline=269]{src/1174089.py}

    \item Buatlah fungsi dengan inputan variabel NPM, didalamnya melakukan penjum-
    lahan dari seluruh dijit NPM tersebut, wajib menggunakan perulangan dan
    atau kondisi.
    \lstinputlisting[firstline=272, lastline=279]{src/1174089.py}

    \item Buatlah fungsi dengan inputan variabel NPM, didalamnya melakukan melakukan
    perkalian dari seluruh dijit NPM tersebut, wajib menggunakan perulangan dan
    atau kondisi.
    \lstinputlisting[firstline=281, lastline=288]{src/1174089.py}

    \item Buatlah fungsi dengan inputan variabel NPM, Lakukan print NPM anda tapi
    hanya dijit genap saja. wajib menggunakan perulangan dan atau kondisi.
    \lstinputlisting[firstline=290, lastline=296]{src/1174089.py}

    \item Buatlah fungsi dengan inputan variabel NPM, Lakukan print NPM anda tapi
    hanya dijit ganjil saja. wajib menggunakan perulangan dan atau kondisi.
    \lstinputlisting[firstline=298, lastline=304]{src/1174089.py}

    \item Buatlah fungsi dengan inputan variabel NPM, Lakukan print NPM anda tapi
    hanya dijit yang termasuk bilangan prima saja. wajib menggunakan perulangan
    dan atau kondisi.
    \lstinputlisting[firstline=306, lastline=320]{src/1174089.py}

    \item Buatlah satu library yang berisi fungsi-fungsi dari nomor diatas dengan nama
    le 3lib.py dan berikan contoh cara pemanggilannya pada le main.py.
    \lstinputlisting[firstline=7, lastline=7]{src/main_advent.py}

    \item Buatlah satu library class dengan nama le kelas3lib.py yang merupakan mod-
    ikasi dari fungsi-fungsi nomor diatas dan berikan contoh cara pemanggilannya
    pada le main.py.
    \lstinputlisting[firstline=8, lastline=9]{src/main_advent.py}
    
\end{enumerate}
\subsubsection{Ketrampilan Penanganan Error}
Error yang di dapat dari mengerjakan tugas ini adalah type error, cara menaggulaginya dengan cara mengecheck kembali codingannya
kemudian run kembali aplikasinya
berikut contoh Penggunaan fungsi try dan exception
\lstinputlisting[firstline=177, lastline=182]{src/1174089.py}
