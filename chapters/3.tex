\section{D. Irga B. Naufal Fakhri}
\subsection{Pemahaman Teori}
\begin{enumerate}
\item Fungsi

Fungsi adalah blok blok kode yang teroorganisir yang dapat digunakan kembali didalam program yang digunakan untuk melakukan suatu perintah yang telah diberikan.
untuk membuat fungsi kita harus menggunakan def kemudian nama fungsinya dan (variable)nya diakhiri oleh tanda :
\lstinputlisting[caption=Contoh kode fungsi inputan ke fungsi., firstline=296, lastline=301]{src/1174066.py}
Fungsi juga berguna untuk melemparkan variable contohnya
\lstinputlisting[caption=Contoh kode fungsi outputan ke fungsi., firstline=303, lastline=308]{src/1174066.py}

\item Paket(Package) atau Libary

Paket atau yang biasa disebut dengan library adalah kumpulan kode-kode fungsi atau method pada python yang dapat dipanggil kedalam program python yang kita buat. Package berada di file terpisah dari main program
cara memanggil package: Pastikan file package ada didalam folder yang sama lalu ditambah import dengan nama filenya tanpa extensi (.py)
\lstinputlisting[caption=Contoh import package atau library., firstline=311, lastline=314]{src/1174066.py}

\item Kelas (Class), Objek (Object), Atribut (Attribute), dan Method

Kelas(Class) adalah sebuah blueprint(cetakan) dari sebuah objek.
Objek(Object) adalah hasil cetakan dari sebuah kelas(class).
Atribut(Attribute) adalah nilai data yang ada didalam sebuah object.
Method adalah sesuatu yang bisa dilakukan oleh object.

\lstinputlisting[caption=Contoh import package atau library., firstline=316, lastline=328]{src/1174066.py}

\item Cara memanggil library dari instansiasi

Cara memanggilnya:
\begin{itemize}
	\item Pertama kita import filenya
	\item kemudian buat variablenya jika menggunakan variable untuk menampung data
	\item Kemudian panggil nama classnya(file) dan panggil fungsinya
	\item Kemudian menggunakan perintah print untuk menampilkan data
\end{itemize}
\lstinputlisting[caption=Contoh package atau library., firstline=6, lastline=9]{src/fungsi_1174066.py} 
\lstinputlisting[caption=Contoh import package atau library., firstline=331, lastline=336]{src/1174066.py}

\item  Contoh pemakaian paket dengan perintah from kalkulator import Penambahan 

Pemakaian package(paket) dengan perintah from namafilenya import berfungsi untuk memanggil fungsi dari nama filenya
\lstinputlisting[caption=Contoh import package atau library., firstline=339, lastline=344]{src/1174066.py}

\item Jelaskan dengan contoh kode, pemakaian paket fungsi didalam folder

Jika file paket ada didalam folder maka kita harus menambahkan lokasi filenya ada didalam folder apa dengan cara menggunakan namafolder.namafile
\lstinputlisting[caption=Contoh import package atau library didalam folder., firstline=346, lastline=351]{src/1174066.py}

\item Jelaskan dengan contoh kode, pemakaian paket fungsi didalam folder

Jika file paket ada didalam folder maka kita harus menambahkan lokasi filenya ada didalam folder apa dengan cara menggunakan namafolder.namafile
\lstinputlisting[caption=Contoh import package atau library didalam folder., firstline=346, lastline=351]{src/1174066.py}
\end{enumerate}

\subsection{Keterampilan Pemograman}
\begin{enumerate}
\item Jawaban nomor 1
\lstinputlisting[firstline=355, lastline=373]{src/1174066.py}

\item Jawaban nomor 2
\lstinputlisting[firstline=377, lastline=384]{src/1174066.py}

\item Jawaban nomor 3
\lstinputlisting[firstline=386, lastline=394]{src/1174066.py}

\item Jawaban nomor 4
\lstinputlisting[firstline=396, lastline=400]{src/1174066.py}

\item Jawaban nomor 5
\lstinputlisting[firstline=402, lastline=405]{src/1174066.py}

\item Jawaban nomor 6
\lstinputlisting[firstline=410, lastline=418]{src/1174066.py}

\item Jawaban nomor 7
\lstinputlisting[firstline=420, lastline=428]{src/1174066.py}

\item Jawaban nomor 8
\lstinputlisting[firstline=431, lastline=439]{src/1174066.py}

\item Jawaban nomor 9
\lstinputlisting[firstline=441, lastline=448]{src/1174066.py}

\item Jawaban nomor 10
\lstinputlisting[firstline=451, lastline=468]{src/1174066.py}

\item Jawaban nomor 11
\lstinputlisting[firstline=8, lastline=20]{src/main_1174066.py}

\item Jawaban nomor 12
\lstinputlisting[firstline=23, lastline=37]{src/main_1174066.py}
\end{enumerate}

\subsection{Ketrampilan Penanganan Error}
\begin{itemize}
\item Syntax Errors

Syntax Errors adalah kesalahan pada penulisan syntax atau kode. Solusinya adalah memperbaiki penulisan syntax atau kode

\item Zero Division Error

ZeroDivisonError adalah exceptions yang terjadi saat eksekusi program menghasilkan perhitungan matematika pembagian dengan angka nol (0). Solusinya adalah tidak membagi suatu yang hasilnya nol.

\item Name Error

NameError adalah exception saat kode melakukan eksekusi terhadap local name atau global name yang tidak terdefinisi atau tidak ada. Solusinya adalah memastikan variabel atau function yang akan dipanggil ada didalam program atau tidak salah mengetikannya.

\item Type Error

TypeError adalah exception saat melakukan eksekusi terhadap suatu operasi atau fungsi dengan type object yang tidak sesuai. Solusinya adalah mengkoversi varibelnya sesuai dengan tipe data sesuai dengan yang akan digunakan.
\end{itemize}
\lstinputlisting[firstline=23, lastline=37]{src/main_1174066.py}