\section{Teori}
\begin{enumerate}

\item Aturan Penulisan Variabel
\begin{flushleft}
\qquad Nama variable bersifat case-sensitive, huruf besar kecil sangat berpengaruh dan nama variable bisa menggunakan \verb|(_)| diawal dan ditengah kalimat. nama variable tidak boleh menggunakan syntax yang ada di python seperti if, else, print dll.\\
\end{flushleft}
Contoh penamaan variable:
	\begin{itemize}
	\item name12
	\item name\verb|_|variable
	\item \verb|_|variable1
	\item nameVarialble
	\item NameVariable
	\end{itemize}
	
Contoh sytanx :
\lstinputlisting[firstline=2, lastline=2]{src/1174067_teori.py}


\item Meminta input dari user dan menampilkannya
\begin{flushleft}
\qquad di python3 menangkap inputan pada keyboard cukup menggunakan fungsi input() untuk mengambil angka atau text. dan untuk outputnya menggunakan fungsi print().\\
\end{flushleft}
contoh syntax :
\lstinputlisting[firstline=5, lastline=6]{src/1174067_teori.py}


\item penggunaan aritmatika dan convert type data
\begin{flushleft}
\qquad aritmatika sering dipakai dalam pemrograman terutama di python. //
\end{flushleft}
Contoh aritmatika:
\lstinputlisting[firstline=9, lastline=63]{src/1174067_teori.py}


\item Perulangan 
\begin{flushleft}
\qquad perulanagan di python bisa menggunakan "for" dan "while". for dipakai untuk perulangan terhitung, sedangkan while dipakai untuk perulangan tidak terhitung.\\
\end{flushleft}
\lstinputlisting[firstline=66, lastline=75]{src/1174067_teori.py}


\item kondisi, kondisi didalam kondisi 
\begin{flushleft}
\qquad kondisi di python menggunakan if dan else.
\end{flushleft}
\lstinputlisting[firstline=78, lastline=101]{src/1174067_teori.py}


\item error yang sering ditemu di python
\begin{flushleft}
\qquad error yang biasanya di temui ada dalam penulisan sytax. Pengurutan sytax jika menggunakan 'space' gunakan 'space' semua, jangan di acak misal menggukana 'tab', jika menggunakan 'tab' semua. cara mengatasinya dengan membaca tutorial penggunaan python sesuai versinya\\
\end{flushleft}

\item Try Execept
\begin{flushleft}
\qquad try execept merupakan cara untuk mengatasi error selain menggukana if else. try execept mempuyai kelebihan dapat memblok sytax yang error.\\
\end{flushleft}
\lstinputlisting[firstline=104, lastline=115]{src/1174067_teori.py}


\end{enumerate}

\section{Praktek}
\begin{enumerate}
\item jawaban
\lstinputlisting[firstline=2, lastline=11]{src/1174067_praktek.py}

\item jawaban
\lstinputlisting[firstline=13, lastline=18]{src/1174067_praktek.py}

\item jawaban
\lstinputlisting[firstline=20, lastline=25]{src/1174067_praktek.py}

\item jawaban
\lstinputlisting[firstline=27, lastline=29]{src/1174067_praktek.py}

\item jawaban
\lstinputlisting[firstline=31, lastline=43]{src/1174067_praktek.py}

\item jawaban
\lstinputlisting[firstline=45, lastline=45]{src/1174067_praktek.py}

\item jawaban
\lstinputlisting[firstline=48, lastline=48]{src/1174067_praktek.py}

\item jawaban
\lstinputlisting[firstline=51, lastline=53]{src/1174067_praktek.py}

\item jawaban
\lstinputlisting[firstline=55, lastline=59]{src/1174067_praktek.py}

\item jawaban
\lstinputlisting[firstline=62, lastline=65]{src/1174067_praktek.py}

\item jawaban
\lstinputlisting[firstline=69, lastline=69]{src/1174067_praktek.py}

\end{enumerate}

\section{Praktek}
\begin{enumerate}
\item syntax error, penulisan syntax salah, cek kembali syntaxnya. division by zero, tidak membagi angka 0.
\item \lstinputlisting[firstline=1, lastline=8]{src/1174067_2err.py}
\end{enumerate}