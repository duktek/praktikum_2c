
\section{Chandra Kirana Poetra}



 
 
\subsection{Sejarah python}

\begin{flushleft}
\qquad Python merupakan suatu bahasa pemrograman yang terinspirasi dari bahasa pemrograman ABC, bahasa pemrograman ABC inilah yang memengaruhi design dan juga pengembangan dari python. Dibuat oleh Guido Van Rossum pada tahun 1989 , python pada awalnya dikembangkan pada tahun 1980an pada saat Guido bekerja di CWI (Centrum voork Wiskunde en Informatica) sebagai programmer yang mengimplementasikan bahasa pemrograman bernama ABC, di sana dia mulai mencari bahasa seperti ABC tapi dengan akses mirip seperti AMOEBA, jadi Guido membuat sendiri bahasa pemrograman sederhana yang bisa menutup celah di ABC. Dan akhirnya pada tahun 1991, versi pertama dari python release ke publik
\end{flushleft}

\subsection{Instalasi Anaconda}
\begin{enumerate}
\item Pertama anda perlu mendownload terlebih dahulu anacondanya.
\item Visi link ini di https://www.anaconda.com/distribution/download-section
\item Setelah download anda selesai, buka file yang anda download tadi
\item Klik next
\begin{figure}[H]
\centering
\includegraphics[width=6cm,height=6cm]{figures/1.png}
\caption{Klik Next}
\label{akhir}
\end{figure}
\item Klik I Agree
\begin{figure}[H]
\centering
\includegraphics[width=6cm,height=6cm]{figures/2.png}
\caption{Klik I Agree}
\label{akhir}
\end{figure}
\item Pilih Just me dan klik next
\begin{figure}[H]
\centering
\includegraphics[width=6cm,height=6cm]{figures/3.png}
\caption{Pilih just me saja}
\label{akhir}
\end{figure}
\item Pilih directory tempat anaconda akan diinstal lalu next
\begin{figure}[H]
\centering
\includegraphics[width=6cm,height=6cm]{figures/4.png}
\caption{Directory tempat anaconda akan diinstalt}
\label{akhir}
\end{figure}
\item Pilih hanya opsi yang bawah saja
\begin{figure}[!htbp]
\centering
\includegraphics[width=6cm,height=6cm]{figures/5.png}
\caption{Opsi Register}
\label{akhir}
\end{figure}
\item Tunggu hingga proses selesai lalu next
\begin{figure}[H]
\centering
\includegraphics[width=6cm,height=6cm]{figures/6.png}
\caption{Tunggu hingga selesai}
\label{akhir}
\end{figure}
\item Opsi tambahan untuk instal visual studio code, skip saja
\begin{figure}[H]
\centering
\includegraphics[width=6cm,height=6cm]{figures/7.png}
\caption{Opsi Tambahan}
\label{akhir}
\end{figure}
\item Klik saja finish
\begin{figure}[H]
\centering
\includegraphics[width=6cm,height=6cm]{figures/7.png}
\caption{Finish}
\label{akhir}
\end{figure}
\end{enumerate}

\subsection{Spyder}
\begin{enumerate}
\item Setelah tadi install anaconda, buka aplikasinya.
\item Biasanya, spyder sudah terinstall bersamaan dengan anaconda, klik launch pada spyder
\begin{figure}[H]
\centering
\includegraphics[width=6cm,height=6cm]{figures/9.png}
\caption{Tampilan awal anaconda}
\label{akhir}
\end{figure}
\item ketika di menu kiri, print("Hellow World") untuk percobaan pertama lalu klik simbo panah hijau untuk run, maka anda akan melihat hasilnya di console
\begin{figure}[H]
\centering
\includegraphics[width=6cm,height=6cm]{figures/10.png}
\caption{IDE Spyder}
\label{akhir}
\end{figure}

\item Output akan dihasilkan di sini
\begin{figure}[H]
\centering
\includegraphics[width=6cm,height=6cm]{figures/11.png}
\caption{Menu Console}
\label{akhir}
\end{figure}


\end{enumerate}


%%%%%%%%%%%%%%%%%%%%%%%%%

\section{Advent Nopele Olansi Damiahan Sihite}
\subsection{Sejarah Phyton}
Phyton diciptakan oleh Guido van Rossum untuk pertama kalinya di Scitchting Mathematisch Centrum (CWI) di Belanda pada awal 1990-an. Bahasa Python terinspirasi oleh bahasa pemrograman ABC. Sampai sekarang, Guido masih menjadi penulis utama untuk Python, meskipun open source terbuka untuk ribuan orang yang juga berkontribusi pada pengembangannya.
\par
Pada tahun 1995, Guido terus membuat python di Corporate for National Research Initiative (CNRI) di Virginia America, tempat ia merilis beberapa versi Phyton.
\par
Pada bulan Mei 2000, Guido dan tim Phyton pindah ke BeOpen.com dan membentuk tim BeOpen PhytonLabs. Pada bulan Oktober tahun yang sama, tim Phyton pindah ke Digital Creation (sekarang Zope Company). Pada tahun 2001, Organisasi Phyton dibentuk, Yayasan Perangkat Lunak Python (PSF). PSF adalah organisasi nirlaba yang khusus dibuat untuk semua hal yang berkaitan dengan kekayaan intelektual Phyton. Perusahaan Zope adalah anggota sponsor PSF.
\par
Semua versi Phyton yang dirilis adalah open source. Dalam sejarahnya, hampir semua rilis pyhton menggunakan lisensi yang kompatibel dengan GFL. Berikutnya adalah versi minor dari walikota dan phyton bersama dengan tanggal rilis.Instalasi anaconda
\subsection{Instalasi Anaconda}
\begin{enumerate}
    \item Pastikan anda mendownload python, sebelum anaconda diinstal
    \item Buka installer klik Next
     \begin{figure}[!htbp]
        \centering
        \includegraphics[width=6cm,height=6cm]{figures/advent/1.png}
        \caption{Tampilan Awal}
        \label{awal}
        \end{figure}
    \item Klik I Agree untuk membuka lisensi
     \begin{figure}[!htbp]
        \centering
        \includegraphics[width=6cm,height=6cm]{figures/advent/2.png}
        \caption{License Agreement}
        \label{awal}
        \end{figure}
    \item  Pilih pada siapa aplikasi diinstal bisa just me dan juga bisa all users
     \begin{figure}[!htbp]
        \centering
        \includegraphics[width=6cm,height=6cm]{figures/advent/3.png}
        \caption{Select Installation Type}
        \label{awal}
        \end{figure}
    \item  Pilih lokasi installasi
    \begin{figure}[!htbp]
        \centering
        \includegraphics[width=6cm,height=6cm]{figures/advent/4.png}
        \caption{Choose Install Location}
        \label{awal}
        \end{figure}
    \item Pilih register anaconda karna add aconda environment tidak remomended
    \begin{figure}[!htbp]
        \centering
        \includegraphics[width=6cm,height=6cm]{figures/advent/5.png}
        \caption{Advanced Intallation Option}
        \label{awal}
        \end{figure}
    \item Tunggu sampai selesai
    \begin{figure}[!htbp]
        \centering
        \includegraphics[width=6cm,height=6cm]{figures/advent/6.png}
        \caption{Installing}
        \label{awal}
        \end{figure}
    \item Klik skip
    \begin{figure}[!htbp]
        \centering
        \includegraphics[width=6cm,height=6cm]{figures/advent/8.png}
        \caption{Anaconda3}
        \label{awal}
        \end{figure}
    \item  Anaconda berhasil di install
    \begin{figure}[!htbp]
        \centering
        \includegraphics[width=6cm,height=6cm]{figures/advent/9.png}
        \caption{Finish Installation}
        \label{awal}
        \end{figure}
\end{enumerate}
\subsection{Menggunakan Spyder}
setelah selesai melakukan installasi anaconda,  maka ada beberapa tool yang digunakan seperti spyder
\begin{figure}[!htbp]
        \centering
        \includegraphics[width=6cm,height=6cm]{figures/advent/10.png}
        \caption{Proses penggunaan spyder}
        \label{awal}
        \end{figure}

Gambar diatas menjelaskan tentang tampilan spyder dan hasil eksekusi program aa


