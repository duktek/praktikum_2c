\begin{document}

\section{Resume}
\begin{flushleft}
\qquad Python dibuat oleh Guido van Rossum yang dirilis pertama  kali pada tahun 1992, 'python 1.0' dirilis pada januari 1994 dan versi terakhir yang ini dirilis saat ini adalah 'Python 3.7' pada 27 Juni 2018. Nama Python sendiri diambil dari acara televisi Monty Python's Flying Circus. Python mendukung multi paradigma pemrograman. pengembangan Python masih dilakukan oleh Team Guido dan Python Software Foundation sebagai pemegang hak cipta intelektual Python sejak versi 2.1 . perbedaan antara python2 dan python3 contohnya pada statement print, pada python2 : 

print "Hello World"

sedangkan pada python3 :

print ("Hello World")
\end{flushleft}

\section{Instalasi}
\subsection{Install Anacaconda}
\begin{enumerate}
\item Download Anaconda3 https://www.anaconda.com/distribution/#linux
\item 'bash Anaconda3-2018.12-Linux-x86-64.sh
\item review license agreement
\item agree license agreement, text 'yes'
\item lokasi install default di /root/anaconda3
\item tambahkan lokasi di /root/.bashrc, text 'yes'
\item selesai
\end{enumerate}

\subsection{Install Python}
\begin{enumerate}
\item apt-get install python3
\item selesai
\end{enumerate}

\subsection{Install Python pip}
\begin{enumerate}
\item apt-get install python3-pip
\item selesai
\end{enumerate}


\end{document}