\documentclass{article}
\usepackage[utf8]{inputenc}
\begin{document}
Ilham Muhammad Ariq 
\par
D4TI2C
\par
1174087
\section{Mengenal Python dan Anaconda}
\begin{enumerate}
    \item 
{SEJARAH PYTHON}
\par
{Pyhton dikembangkan pada tahun 1990 oleh Guido van Rossum di CWI Amsterdam sebagai kelanjutan dari bahasa pemrograman ABC.}
\par
{Tahun 1995, Guido pindah ke CNRI di Virginia Amerika sambil terus melanjutkan pengembangan Python. Versi terakhir yang dikeluarkan adalah 1.6. Tahun 2000, Guido dan para pengembang inti Python pindah ke BeOpen.com yang merupakan sebuah perusahaan komersial dan membentuk BeOpen PythonLabs. Python 2.0 dikeluarkan oleh BeOpen. Setelah mengeluarkan Python 2.0, Guido dan beberapa anggota tim PythonLabs pindah ke DigitalCreations.
\par
Saat ini pengembangan Python terus dilakukan oleh sekumpulan pemrogram yang dikoordinir Guido dan Python Software Foundation. Python Software Foundation adalah sebuah organisasi non-profit yang dibentuk sebagai pemegang hak cipta intelektual Python sejak versi 2.1 dan dengan demikian mencegah Python dimiliki oleh perusahaan komersial. Saat ini distribusi Python sudah mencapai versi 2.7.14 dan versi 3.6.3
\par
Nama Python dipilih oleh Guido sebagai nama bahasa ciptaannya karena kecintaan Guido pada acara televisi Monty Python's Flying Circus. 
    \item
PERBEDAAN PYTHON 2 DAN 3
\par
Python versi 2 merupakan versi yang banyak digunakan saat ini, baik dilingkungan produksi dan pengembangan.Sementara Python versi 3 adalah pengembangan lanjutan dari versi 2. Python 3 memiliki lebih banyak fitur dibandingkan Python 2. Untuk membuka Python 2 kita hanya menggunakan perintah python saja, sedangkan Python 3 menggunakan perintah python3.
 }   
\end{enumerate}

\section{Cara Pemakaian Python}
\par
Untuk menuliskan script Python, cukup buka terminal yang ingin digunakan misalkan cmd dan ketikan pyhton
contoh syntax dasar Hello word
\par
ketikan :
\par
print('hello word')
\par
outputnya :
\par
hello word

\section{Instalasi Python}
\begin{enumerate}
\item
Download file python terlebih dahulu
\item
Kemudian install file yg telah didownload
\item
Pilih saja ‘Install for all users’ agar bisa dipakai untuk semua user di komputernya
\item
Tentukan lokasi python akan diinstal, kemudian klik next.
\item
Pada tahapan ini, kita akan menentukan fitur-fitur yang akan diinstal.
Jangan lupa untuk mengaktifkan ‘Add python.exe to path’ agar perintah python dikenali pada CMD (Command Prompt).
\end{enumerate}

\section{Indentasi}
\par
Python memanfaatkan indentasi untuk membuka/menutup fungsi tersebut. jika melakukan coding dengan notepad++. Pada notepad++ setting perintah Tab menjadi indentasi 4 karakter spasi , dengan memilih Setting -> Preferences ceklist box Replace by Space dengan Tab Size = 4.Dengan begitu yang tinggal mengklik ‘Tab‘ pada keyboard untuk melakukan indentasi kedalam, tanpa harus  mengisi dengan spasi sebanyak 4 kali.
\end{document}

