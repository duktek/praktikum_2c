\documentclass{article}
\usepackage[utf8]{inputenc}

\title{PEMROGRAMAN III}
\author{HANDI HERMAWAN }


\begin{document}

\maketitle

\section{Chapter 1 }
\subsection{BACKGROUND}
	Phyton adalah bahasa pemrograman dinamis yang mendukung pemrograman berorientasi obyek. Python dapat digunakan untuk berbagai keperluan pengembangan perangkat lunak dan dapat berjalan di berbagai platform sistem operasi. Python sendiri merupakan bahasa pemrograman yang dapat digunakan dengan berbagai paradigma. Mulai dari scripting sederhana hingga object oriented sehingga sangat cocok untuk penggunaan sehari – hari. Python sendiri dipakai diberbagai industri, seperti misalnya industri aeronautica, bahkan studio besar seperti Industrial Light & Magic yang notabene subsidiari dari Lucasfilm (Disney) – penggarap VFX film besar seperti Avenger dan sekuel Star Wars. Sebagian besar industri animasi terlebih pengguna Autodesk Maya memang menggunakan Python untuk membantu pekerjaan. Python menawarkan potensi yang luar biasa. Python dikembangkan pertama kali oleh Guido van Rossum pada akhir tahun 1980 dan dipublikasikan pertama kali pada 1991. Disebut sebagai penerus pemrograman ABC, Python pada awal rilis sudah disertai dengan fungsi seperti exception, function bahkan class dengan inheritance. Ketika itu sebuah Usernet newsgroup bernama comp.lang.python dibuat di 1994, grup ini membentuk langkah awal untuk Python sebagai salah satu bahasa pemrograman yang sangat populer untuk pengembangan Open Source Software.
	Dipublikasikan pada akhir tahun 2000, Python 2 dinilai lebih transparan dan inklusif untuk pengembangan software ketimbang versi sebelumnya. Hal ini didukung dengan adanya PEP – Python Enhancement Proposal, sebuah spesifikasi teknis yang menjadi tuntunan informasi untuk penggunanya dan menggambarkan fitur baru pada Python itu sendiri.
	Python 2 dilengkapi dengan berbagai fitur programatikal seperti cycle-detecting garbage collector untuk mengotomasi manajemen memori, peningkatan dukungan untuk Unicode, list comprehension untuk membuat sebuah list berdasarkan list yang sudah ada. Unifikasi pada tipe data Python dan class ke satu hirarki terjadi pada rilis Python 2.2
	Python 3 diharapkan sebagai masa depan Python dan merupakan versi yang saat tulisan ini dibuat masih aktif dikembangkan. Python 3 sendiri adalah versi dengan banyak perubahan yang dirilis akhir tahun 2008. Fokus dari Python 3 itu sendiri adalah untuk melakukan perapian pada codebase dan menghapuskan duplikasi (redundancy). Perubahan terbesar pada Python 3 termasuk memasukkan statemen print ke dalam built-in function.
	Python 3 mengalami hambatan pada pengadopsiannya. Itu akibat dari tidak adanya backwards compatibility dengan Python 2. Hal ini membuat pengguna Python sangat berat hati untuk pindah ke versi 3 ini. Tambahannya, banyak sekali library yang hanya tersedia untuk Python 2., tapi setelah tim pengembangan di balik Python 3 telah berulang kali menjelaskan bahwa dukungan terhadap Python 2 akan segera dihentikan, dan semakin banyak libary disalin ke Python 3, maka penerapan Python 3 semakin lama semakin meningkat.

\subsection{PROBLEM}
masalah yang akan dibahas dalam makalah ini adalah mengenai bahasa pemograman yang meliputi :
\begin{itemize}
\item Bagaimana sejarah dan perkembangan Pemograman Bahasa phyton?
\item Cara Mengisnstal dan menjalankan Program Phyton ?
\item Bentuk Dasar Pada Bahsa Phyton ?
\end{itemize}
\subsection{OBJECTIVE AND CONTRIBUTION}
\item OBJECTIVE
Untuk memahami kegunaan dan dasar pada bahasa Pemograman Phyton juga Python memungkinkan kita untuk membagi-bagi program menjadi modul-modul yang dapat di gunakan kembali dalam program python lainnya.python mempunyai koleksi besar modul-modul standar yang dapat anda gunakan sebagai dasar bagi program atau sebagai contoh untuk awal mempeljari cara memprogram dengan python. terdapat juga modul build-in yang menyediakan Fasilitas,seperti I/O file,system call,socket,dan bahkan antarmka untuk GUI toolkit seperti tkinter.
\item CONTRIBUTION	
untuk berkontribusi di pelajaran python, dimana website-nya sendiri dikelola digithub. untuk berkontribusi pelajaran python, dimana setelah digithub nanti saya telah membuat projek semoga bisa dimanipulasi dan disempurnakan oleh pembaca nanti.
\end{itemize}
\subsection{SCOOP AND ENVIRONTMENT}
Python adalah bahasa pemrograman interpretatif multiguna dengan filosofi perancangan yang berfokus pada tingkat keterbacaan kode. Python diklaim sebagai bahasa yang menggabungkan kapabilitas, kemampuan, dengan sintaksis kode yang sangat jelas, dan dilengkapi dengan fungsionalitas pustaka standar yang besar serta komprehensif. Python mendukung multi paradigma pemrograman, utamanya; namun tidak dibatasi; pada pemrograman berorientasi objek, pemrograman imperatif, dan pemrograman fungsional. Salah satu fitur yang tersedia pada python adalah sebagai bahasa pemrograman dinamis yang dilengkapi dengan manajemen memori otomatis. Seperti halnya pada bahasa pemrograman dinamis lainnya, python umumnya digunakan sebagai bahasa skrip meski pada praktiknya penggunaan bahasa ini lebih luas mencakup konteks pemanfaatan yang umumnya tidak dilakukan dengan menggunakan bahasa skrip. Python dapat digunakan untuk berbagai keperluan pengembangan perangkat lunak dan dapat berjalan di berbagai platform sistem operasi.
\end{document}
